%!TEX root = ../template.tex
%%%%%%%%%%%%%%%%%%%%%%%%%%%%%%%%%%%%%%%%%%%%%%%%%%%%%%%%%%%%%%%%%%%
%% chapter1.tex
%% NOVA thesis document file
%%
%% Chapter with introduction
%%%%%%%%%%%%%%%%%%%%%%%%%%%%%%%%%%%%%%%%%%%%%%%%%%%%%%%%%%%%%%%%%%%

\typeout{NT FILE chapter1.tex}%

\chapter{Introdução}
\label{cha:introdução}

\noindent This is the \gls{novathesis} \LaTeX\ template \ntindex[Template!]{Version} \novathesisversion\ from   {Template!date}\novathesisdate.

\section{Enquadramento e Motivação}
\label{sec:Enquadramento_e_Motivação}

Com o aumento da quantidade de veículos nas estradas e, em anos mais recentes, o aumento da autonomia dos mesmos,
é necessário que sejam implementadas melhorias ao nível da segurança dos peões através do aproveitamento das
comunicações no contexto de redes veiculares [1].
\par{Enquanto que, ao nível dos veículos, já existem equipamentos, infraestrutura, normas, e tecnologias especializadas que
usam as comunicações para melhorar a eficiência do tráfego e a segurança dos veículos, o mesmo não acontece ao nível
dos peões (e outros utilizadores das vias públicas) [2]. Por este motivo é necessário encontrar métodos que facilitem e
melhorem a capacidade destes intervenientes, transmitirem, com fiabilidade, informações acerca da sua posição e/ou
movimentação. O recurso às comunicações V2X, em conjunto com o aumento da capacidade sensorial do veículo pode
vir a aumentar a segurança de todos os que utilizam a redes rodoviárias [3].}
\par{Tendo em conta o potencial do paradigma da Internet das Coisas como uma possível solução para as necessidades
referidas anteriormente, é relevante avaliar a possibilidade da sua integração nas redes veiculares, pela sua versatilidade,
facilidade de implementação e aumento exponencial das suas capacidades [4].}

\section{Objetivos}
\label{}

\section{Estrutura da Dissertação}
\label{}

