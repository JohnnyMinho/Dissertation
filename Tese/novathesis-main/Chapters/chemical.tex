%!TEX root = ../template.tex
%%%%%%%%%%%%%%%%%%%%%%%%%%%%%%%%%%%%%%%%%%%%%%%%%%%%%%%%%%%%%%%%%%%%
%% chemical.tex
%% NOVA thesis document file
%%
%% Additional list of symbols
%%%%%%%%%%%%%%%%%%%%%%%%%%%%%%%%%%%%%%%%%%%%%%%%%%%%%%%%%%%%%%%%%%%%

\typeout{NT FILE chemical.tex}%

% Define a new gloassary.
%        \newglossary[XXg]{type_name}{XXs}{XXo}{Name to be printed}
\newglossary[chg]{chemical}{chs}{cho}{Chemical Symbols}
% If you use “latexmk” also edit file "latexmkrc" and add the following
% rule at the top of the file.
%        add_cus_dep('XXo', 'XXs', 0, 'makeglo2gls');

\newglossaryentry{chem:potassio}% label
{%
  type=chemical,% glossary type
  name={$K^+$},%
  description={Ião positivo de Potássio},%
  sort={K}%
}

\newglossaryentry{chem:sodio}% label
{%
  type=chemical,% glossary type
  name={$Na^+$},%
  description={Ião positivo de Sódio},%
  sort={Na}%
}
